%%%%%%%%%%%%%%%%%%%%%%%%%%%%%%%%%%%%%%%%%%%%%%%%%%%%%%%%%%%%%%%%%%%%%%%%
% Project: SRT Hardware Manual
% Source: MIT Haystack
% Author: Nickalas Reynolds, John Tobin
% Location: The University of Oklahoma
% Date: Feb 2017
%%%%%%%%%%%%%%%%%%%%%%%%%%%%%%%%%%%%%%%%%%%%%%%%%%%%%%%%%%%%%%%%%%%%%%%%

% functions


% packages
\documentclass[a4paper,10pt]{report}
\usepackage[T1]{fontenc}
\usepackage[utf8]{inputenc}
\usepackage{lmodern}
\usepackage{hyperref}
\usepackage{geometry}
\usepackage{graphicx}
\usepackage{amsmath}
\usepackage[english]{babel}
\geometry{margin=0.5in}
\usepackage{listings}
\usepackage{color}
\usepackage{multicol}

 % colors
\definecolor{codegreen}{rgb}{0,0.6,0}
\definecolor{codegray}{rgb}{0.5,0.5,0.5}
\definecolor{codepurple}{rgb}{0.58,0,0.82}
\definecolor{backcolour}{rgb}{0.95,0.95,0.92}

% coding inline
\lstdefinestyle{mystyle}{
    backgroundcolor=\color{backcolour},   
    commentstyle=\color{codegreen},
    keywordstyle=\color{magenta},
    numberstyle=\tiny\color{codegray},
    stringstyle=\color{codepurple},
    basicstyle=\footnotesize,
    breakatwhitespace=false,         
    breaklines=true,                 
    captionpos=b,                    
    keepspaces=true,                 
    numbers=left,                    
    numbersep=5pt,                  
    showspaces=false,                
    showstringspaces=false,
    showtabs=false,                  
    tabsize=2
} 
\lstset{style=mystyle}
\newcommand{\code}[1]{\colorbox{backcolour}{\text{#1}}}
\newcommand{\git}{\url{https://github.com/OUsrt/srtn.git}}
\newcommand{\ben}{\url{https://github.com/BenningtonCS/Telescope-2014/wiki}}

% Book's title and subtitle
\title{\Huge \textbf{The University of Oklahoma} \\ \textbf{Nielsen Hall SRT}\\ \huge Hardware Manual \\ \huge Version 0.1}
% Author
\author{\textsc{John Tobin\footnote{jjtobin@ou.edu}}\footnote{GitHub: \git , Group Email: \url{nhnradiotelescope@groups.ou.edu}}\\\textsc{Nickalas Reynolds\footnotemark[2]\footnote{nickreynolds@ou.edu}}}


\begin{document}

%\frontmatter
\maketitle


%%%%%%%%%%%%%%%%%%%%%%%%%%%%%%%%%%%%%%%%%%%%%%%%%%%%%%%%%%%%%%%%%%%%%%%%
% Abstract %
%%%%%%%%%%%%%%%%%%%%%%%%%%%%%%%%%%%%%%%%%%%%%%%%%%%%%%%%%%%%%%%%%%%%%%%%
\chapter*{Abstract}
The University of Oklahoma's Small Radio Telescope (3m) was sponsored by Dr. John Tobin as an outreach and educational tool for the Norman Community. The use of the telescope helps to aid in expanding one's knowledge of radio astronomy, engineering principles, and proper research methods. The kit was implemented off of MIT Haystack's SRT, fabricated through RF Ham Design, constructed by a team of students and professors\footnote{Prof: John Tobin; Grad students: Paul Canton, Hyunseop Choi, Nickalas Reynolds, Rajeeb Sharma, G-PSI; Undergrad Students: Jacob Gill, Lisa Patel, and Brian Stephensen; Staff: Barry Bergeron, Andy Feldt, Debi Schoenberger, John Snellings, Adrienne Wade, and Joel Young}, and the code was provided by MIT Haystack under the MIT Public License. 

Information specifically regarding the MIT Haystack Observatory can be found at \url{http://www.haystack.mit.edu/edu/undergrad/srt/index.html}
%%%%%%%%%%%%%%%%%%%%%%%%%%%%%%%%%%%%%%%%%%%%%%%%%%%
\section*{Additional Information}
The website\footnote{\git} for this file contains:
\begin{itemize}
  \item A link to the downloadable PDF and \LaTeX\space code.
  \item The SRT source code.
  \item Useful scripts for observing and parsing the data
\end{itemize}

%%%%%%%%%%%%%%%%%%%%%%%%%%%%%%%%%%%%%%%%%%%%%%%%%%%%%%%%%%%%%%%%%%%%%%%%
% Acknowledgements %
%%%%%%%%%%%%%%%%%%%%%%%%%%%%%%%%%%%%%%%%%%%%%%%%%%%%%%%%%%%%%%%%%%%%%%%%
\section*{Acknowledgements}
\begin{itemize}
\item Hardware design, development of software, and instructions: (\url{www.haystack.mit.edu/edu/undergrad/srt/})
\item Fabrication of the SRT parts: (\url{https://www.rfhamdesign.com} )
\item Parse code and instructions: (\ben)
\end{itemize}

%%%%%%%%%%%%%%%%%%%%%%%%%%%%%%%%%%%%%%%%%%%%%%%%%%%%%%%%%%%%%%%%%%%%%%%%
% Auto-generated table of contents, list of figures and list of tables %
%%%%%%%%%%%%%%%%%%%%%%%%%%%%%%%%%%%%%%%%%%%%%%%%%%%%%%%%%%%%%%%%%%%%%%%%
\tableofcontents


%%%%%%%%%%%%%%%%%%%%%%%%%%%%%%%%%%%%%%%%%%%%%%%%%%%%%%%%%%%%%%%%%%%%%%%%
% Hardware %
%%%%%%%%%%%%%%%%%%%%%%%%%%%%%%%%%%%%%%%%%%%%%%%%%%%%%%%%%%%%%%%%%%%%%%%%
\chapter{Building the Hardware}
The tools needed were: protective gloves, protective goggles, electric hand drill, \(\frac{1}{4}\) in drill bit, hammer, mallet, riveter, rivets, wire cutter, sharpie, electric cut-off tool (metal blade), soldering iron, mini-pliers, wrench, and a bubble level.
%%%%%%%%%%%%%%%%%%%%%%%%%%%%%%%%%%%%%%%%%%%%%%%%%%%
\section{Rotor}
*Working on this section

%%%%%%%%%%%%%%%%%%%%%%%%%%%%%%%%%%%%%%%%%%%%%%%%%%%
\section{LNA}
The low-noise amplifier is located in an environmental resistant box mounted on the rear of the horn, minimizing the blockage of the aperture. The LNA consists of two ultra-low-noise amplifier modules, a band-pass filter, and a bias-tee to power the amplifiers. Two holes will be made in the aluminum case. 
% Picture of lna case
\begin{itemize}
\item Drill the two holes with a $\frac{1}{4}$'' drill bit.
\item Remove the nut/washer from one of the SMA-F to SMA-F bulkhead connectors
\item Secure the O-ring against the flange and insert the longer side of the connector into the case and secure it with the washer/nut, using a wrench.
\item Connect the filter to the ``out'' port of the ZX60-1614LN-S amplifier and the SMA-M to SMA-M adapter to the ``in'' port. Insert the 3'' SMA cable to the other end of the filter
\item Put this total assembly into the box and connect the adapter to the connector
\item Attach the other end of the SMA cable to the ``in'' port of the second amplifier.
\item Section two pieces of 12V wire and tin the ends of them and the 12V pins of the amplifiers
\item Connect the two 12V pins with one wire and attach the other to the 12V pin of the amplifier not connected to the adapter
\end{itemize}
% picture of inside lna case
\begin{itemize}
\item Attach the 3'' SMA cable to the ``out'' port of the second amplifier and to the ``RF'' port of the bias-tee. 
\item Connect the other SMA-F to SMA-F connector and insert it through the second hole
\item Connect the ``RF\&DC'' port of the bias-tee to the connector
\item Tin the 12V pin of the bias-tee and solder the remaining 12V wire to this pun
\end{itemize}
% Picture of complete setup lna case
\begin{itemize}
\item Place the rubber gasket lid on the case and screw it in.
\end{itemize}

%%%%%%%%%%%%%%%%%%%%%%%%%%%%%%%%%%%%%%%%%%%%%%%%%%%
\section{Feed}
This apparatus consists of a helical antenna inside an aluminum cavity or horn. The helix is a copper tape stretched around the polystyrene foam cylinder, with a folded foil plate to match impedance. The feed is constructed from a 2.5in Styrofoam cylinder. The height of the cylinder is determined by the desired wavelength and is $\frac{1}{4}$ wavelength.
% picture of feed schematic
\begin{itemize}
\item Cut the cake pan as indicated on the drawing of the feed.
\item Cut the foam rod to 73.7mm in length and drill a $\frac{1}{4}$ '' hole down the long axis of the rod. 
\item 1mm from the edge of the rod, draw a mark. Make two more marks vertically above this mark each 30mm above the previous mark
\item Cut out a strop of the copper tape 4x439mm. Tape one end at the lowest mark and without taping the rest wrap the strip around the foam such that the lower edge of the strip is flush between the previous two marks.
\item With a ruler, verify the helix has vertical spacing of 30mm. Once you get the strip in a good position, trace the strips upper edge.
\item make a mark 15mm from the lower end of the strip and bend the foil back. Make another mark 13mm from the first mark and then a final mark 26.25mm past that
\item Remove the adhesive backing from the bent foil and fold the copper onto itself. Place the bend on the lowest mark on the foam and apply the foil to the foam. Keep the helix aligned with the previously designated points.
\end{itemize}
% picture of foam with wrap
\begin{itemize}
\item Cut out a strip of foil 26.25x13mm and leave and additional 3-4mm wide strip on one of the long edges. This will be the basis to form the impedance matching section.
\item Fold this strip so it is 90 degrees to the rest of the foil. Center it on the helix between the two marks that are 26.25 mm apart and solder the strip to the helix.
\item Solder the bent over power of the end of the foil to the SMA connector.
\item Place a $\frac{1}{4}$'' O-ring to the SMA connector and insert into the cake pan center hole. \item Section out a 63mm square piece of PC board and drill a $\frac{1}{4}$'' hole in the center and a $\frac{1}{4}$'' indent as indicated.
\end{itemize}
% picture of PC board
\begin{itemize}
\item Run the bolt through the center of the foam, board, and the cake pan and secure it with hand tightened nut and washer. Make sure the soldered end of the SMA connector is against the bottom of the cake pan and not propped by the PC board.
\item Test the apparatus with a network analyzer by attaching the network analyzer to the SMA connect and set to measure S11. Should read roughly -20 dB.
\item A shorted $\frac{1}{4}$ wavelength stub should be added to the feed to act as a DC path for lightning and static protection for the amplifiers
\begin{itemize}
\item Cut out 42mm of long piece coaxial cable
\item Strip the shield and insulation away $\sim$2mm on one end and 4mm on the other.
\item Bend over the 4mm end and solder it to the shield
\item Solder the 2mm side to the center connector of the stub in the SMA connector and solder the shield to one of the ground pins of the SMA connector
\end{itemize}
\end{itemize}
%picture of stub
\begin{itemize}
\item The LNA mount plate will now be made. Drill the holes as indicated by the diagram %ref to previous digram
\item Secure the plate to the back of the pan with four machine screws and nuts through the holes in the corners. Tighten with a wrench
\item Thread a nut onto each of the 1'' machine screws and place a lock washer on top. Place the LNA onto the machine screws and connect the port of the LNA to the SMA connector on the feed with a SMA-M to SMA-M right angle connector
\item Using the nuts, level the LNA case on the pan.
\item thread the final nuts onto the machine screws and tighten with a wrench.
\item Attach the quadrapod legs to the feed
\end{itemize}
% picture of final setup

%%%%%%%%%%%%%%%%%%%%%%%%%%%%%%%%%%%%%%%%%%%%%%%%%%%
\section{Antenna}
%%%%%%%%%%%%%%%%%%%%%%%%%
\subsection{Pier}
The dish was ``portably'' mounted on the roof of Nielsen Hall. There is 1000lbs of ballast holding the pier in place as per wind regulations however, it is not permanently mounted to the building.
The mount has a wide base to support a 3in pier pipe in which the SPID rotor mounts on top of. An adapter is needed to convert the pier out from 3in to 2.875in. The adapter is held in place by 12bolts and re-leveled. Secure the PID rotor to the adapter. The assembly of the mount is fairly straightforward.
%%%%%%%%%%%%%%%%%%%%%%%%%
\subsection{Dish}
The dish is constructed of 6mm galvanized aluminum. The ribs are mounted onto the 2 ``hubs'' (CNC milled centers) with bolts. The hubs are 20cm round and 10mm thick aluminum. Each rib is secured with 2 bolts into the hub. Before you tighten the bolts, make sure the ribs are spaced evenly as they provide the support for the mesh. Fluctuations in the mesh reduce the efficiency of the telescope. 

Feed the aluminum wire through the four co-centric holes cut in the ribs to secure them in place. The wire will need to be cut precisely to size and crimped in place with the female-female inserts included. 

It is suggested at this point to use large sheets of plotter paper to directly map the size of the mesh sheets needed. This will ensure accurate cutting of the mesh. The mesh should be of an appropriate size, completely overlapping both sides of the rib and with $\sim$4in of spare material on the outer rim to allow fastening. Make sure both the cut mesh and the ribs are label to correspond with the appropriate piece. Once all of the mesh needed is cut, it is time to mount them to the dish.

Line the mesh sheets up to the ribs of the telescope. Use the hand drill to score the mesh and drill through the ribs and 2x20mm aluminum strips. You can then use the riveter to secure the mesh onto the ribs. Use about 6 rivets per rib to secure the mesh. Once all of the mesh is secure, remove the excess mesh around the hub using the cut-off tool with the metal attachment. 

Use zip-ties to secure the mesh onto the aluminum wires fed through the ribs. Space them as evenly as possible and make sure they are held secure. The mesh should be as smooth as possible and follow the curvature of the ribs. Also use zip-ties to wrap the excess $\sim$4in of mesh around the 3x20mm aluminum strip on the outer rim and secure it in place. Remove all unnecessary mesh excess with the cut-off tool. 

Use several people to mount the dish onto the rotor using the provided bracket and bolts. The dish needs to be securely fastened to the rotor and centered. Use the mallet to ``ease'' it into its final resting position. The dish itself isn't extremely heavy but acts as a giant sail and can be very difficult to secure in place.

Mount the quadrapod legs into the dish. You will have to play around with the placement of the legs such that the focus of the dish is on the axis of the helix about half way along its length. You can use string to measure the true diameter and the measure the distance from the string to the center of the dish. The focus is given by this formula:
\begin{align}
    f=\frac{D^2}{16\times d}
\end{align}
Once the focal length is found and the feed is mounted, use coaxial cable to connect the feed to the receiver. Run the wire down one of the legs and secure it with zip-ties. Secure it along the rib on the back of the telescope and bundle it with the other wires. These wires should be bundled carefully and should feed into protective housing such that environmental effects don't interfere and such that the telescope has full range along its slew without disrupting the bundle. 


%%%%%%%%%%%%%%%%%%%%%%%%%%%%%%%%%%%%%%%%%%%%%%%%%%%%%%%%%%%%%%%%%%%%%%%%
% Parts List %
%%%%%%%%%%%%%%%%%%%%%%%%%%%%%%%%%%%%%%%%%%%%%%%%%%%%%%%%%%%%%%%%%%%%%%%%
\chapter{Complete Parts List}
%%%%%%%%%%%%%%%%%%%%%%%%%%%%%%%%%%%%%%%%%%%%%%%%%%%
\section{Rotor}
\begin{align}
\text{Power supply unit: PW15015LC} & \nonumber \\
\text{Specifications:}  & \nonumber \\
 & \text{Voltage adj. range: 13.5 - 16.5 Volt / 10 Amp DC} \nonumber \\
 & \text{Overload protection} \nonumber \\
 & \text{Rimpel noise max 180 mVpp} \nonumber \\
 & \text{AC input range: 88 $\sim$ 132VAC/176 $\sim$ 264VAC selected switch} \nonumber \\
 & \text{Dimension: 19 x 11 x 5 cm (LxBxH)} \nonumber \\
 & \text{Weight: 0.8 kg} \nonumber \\
\text{Power supply unit: PW32015} & \nonumber \\
\text{Specifications:} & \nonumber \\
 & \text{Universal AC input / Full range} \nonumber \\
 & \text{Built-in active PFC function. PF>0.95} \nonumber \\
 & \text{Voltage adj. range: 13.5 - 18 Volt / 20 Amp DC} \nonumber \\
 & \text{Protections: Short circuit/Over load/Over voltage/Over temperature} \nonumber \\
 & \text{Forced air cooling by built-in DC Fan} \nonumber \\
 & \text{Built-in fan speed control} \nonumber \\
 & \text{Voltage range: 88 $\sim$ 264VAC 124 $\sim$ 370VDC} \nonumber \\
 & \text{Dimensions: 23 x 11.5 x 5.5 cm (LxBxH)} \nonumber \\
 & \text{Weight: 1.1 Kg} \nonumber \\
 & \text{Core control cable} \nonumber \\
\text{SPID RAS rotator bracket: FPD-BR02} & \nonumber 
\end{align}

%%%%%%%%%%%%%%%%%%%%%%%%%%%%%%%%%%%%%%%%%%%%%%%%%%%
\section{Mount \& Dish}
\begin{multicols}{2}
\begin{itemize}
\item Mount Plate/bracket
\item 3" x 10' Pier
\item Ballast: 1000lbs
\item 6mm galvanized aluminum wire mesh 4x150ft
\item 2 pcs CNC milled centers (Hub)
\item 12 Rib's made of 20mm sq aluminium tube
\item 4 inner rings:2mm x 100ft are used to secure mesh
\item Outer rim made of 3x20mm aluminum strip
\item Mesh secured by using of 2x20mm aluminum strip
\item 3-Leg feed support 15 or 20 square aluminum tube
\item 3" to 2.875" adapter for rotor
\end{itemize}
\end{multicols}

%%%%%%%%%%%%%%%%%%%%%%%%%%%%%%%%%%%%%%%%%%%%%%%%%%%
\clearpage
\section{Feed \& LNA}
\begin{multicols}{2}
\begin{itemize}
\item Ultra Low Noise Amplifier
\item Coaxial Bias-Tee
\item Coaxial Cable 086-6SM+
\item Coaxial Cable 086-4SM+
\item SMA-F to SMA-F Adapter
\item SMA-F to Solder Pin Bulkhead Connector
\item SMA-M to SMA-M Right Angle Adapter
\item 3M Adhesive Copper Foil Tape
\item 22 Gauge Copper Wire
\item Semi-Rigid Coaxial Cable
\item F-Type Male to BNC Female
\item SMA-M to SMA-M Coupler
\item DC Barrel Jack connector
\item F-type plug connector
\item Cap Screw 1/4-20x3.25"
\item Lock Washers 6-32
\item Machine Screws, 6-32x1/2"
\item Machine Screws, 6-32X1"
\item Nuts 6-32
\item Nuts 1/4-20
\item Washers 1/4"
\item Styrofoam Rod 2.5" x 24"
\item Cake Pan 7"x3"
\item O-rings 7/32" inner diameter
\item O-rings 1/4" inner diameter
\item SMA-M Crimp Connector Straight
\item LMR-400
\item Type-F Male Crimp Connector
\item In-Line Amplifier
\item BNC Female Jack to Type-F Male Plug
\item BNC-M to BNC-M Patch Cable
\item Power Injector
\item BNC-F to SMA-M Adapter
\item Bandpass Filter
\item SMA-F to BNC-F Adapter
\item BNC-F to SMB Plug Adapter
\item AC-to-DC Power Supply
\item Breadboard
\item DC Jack
\item F-Type Plug
\end{itemize}
\end{multicols}

\end{document}
% end of document